\documentclass[11pt]{article} % 11pt font

% Conform to NSF formatting requirements
% see: https://www.nsf.gov/pubs/2021/nsf21602/nsf21602.pdf
% --------------------------------------
% 1in margins
\usepackage[margin=1in]{geometry}

% use times new roman for main text
\usepackage{fontspec}
\setmainfont{Times New Roman}
% use cambria for math
\usepackage{unicode-math}
\setmathfont{[Cambria-Math.ttf]}
% single line spacing
\usepackage{setspace}
\singlespacing
% remove the page numbers
\pagenumbering{gobble}

% To fit more into the proposal, 
% let's make the section titles tiny and compact
\usepackage[tiny,compact]{titlesec}
\titleformat{\section}[runin]{\bfseries}{\thesection}{1em}{}
\titleformat{\subsection}[runin]{\bfseries}{\thesubsection}{1em}{}

% and let's compress the references too
\usepackage[numbers]{natbib}
\renewcommand{\refname}{References \vspace{-0.6\baselineskip}} % no extra vert. space after title
\setlength{\bibsep}{0pt} % no extra vert. space between bib items

% BEGINNING OF THE DOCUMENT
\begin{document}

\begin{center}
\large{\bf Title of the project}
\end{center}

\section*{Outline}

The outlines goes here.

\section*{Motivation and Intellectual Merit}

Here is a citation to one paper \cite{paper1}. Here one to another one \cite{paper2}, and here is one to both papers \cite{paper1,paper2}.

\section*{Research Plan}

Here is an equation: 
\begin{equation}
\label{eq:pythagoras}
a^2 + b^2 = c^2
\end{equation}
See Equation~\ref{eq:pythagoras}. 

\section*{Broader Impacts}


\bibliographystyle{unsrtnat}
\bibliography{references}

\end{document}